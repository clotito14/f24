\documentclass{article}

\usepackage[letterpaper, tmargin=0in]{geometry}
\usepackage{amsmath, amsfonts, amsthm}
\usepackage{enumerate}

\title{Proof of \(v_p(1) = 0\).}
\author{Chase A. Lotito}

\begin{document}

\maketitle{}

\section{Definiton of a (tangent) vector.}

A (tangent) vector at a point \(p \in M\), where \(M\) is a smooth manifold, is an operator:

\begin{equation*}
    v_p : \mathcal{F}M \to \mathbb{R}
\end{equation*}

\noindent
Our vector \(v_p\), the operator, takes a smooth function (which itself maps our manifold to a scalar quantity) and provides a real number, given these conditions are satisfied:

\begin{enumerate}[i.]
    \item \( v_p (f + g) = v_p(f) + v_p(g) \)
    \item \( v_p (cf) = c (v_p(f)), ~~~ c \in \mathbb{R} \)
    \item \( v_p(fg) = v_p(f)g(p) + f(p)v_p(g) \)
\end{enumerate}

\section{Proof of why \(v_p(1) = 0\)}

From the definition of the vector, we see that it acts like a derivative. Both operate on functions and provide a scalar quantity in return. From derivatives, we know that:

\begin{equation*}
    \frac{d}{dx}(1) = 0
    \label{eq:derivative-of-one}
\end{equation*}

\noindent
Can we prove then, that the (tangent) vector operating on unity is also nothing, i.e. \(v_p(1) = 0\)?

\begin{proof}[Proof]

Let our tangent vector \(v_p\) act on the following smooth function from \(\mathcal{F}M\):

\begin{equation*}
    f = x^0 + c^1 x^1 + c^2 x^2 + \cdots + c^n x^n, ~~~ c^i \in \mathbb{R}
\end{equation*}

\noindent
From \(f\), the first term \(x^0 \in \mathbb{R}\). If we take \(f+1\), then we'd have:

\begin{equation*}
    f = 1 + x^0 + c^1 x^1 + c^2 x^2 + \cdots + c^n x^n, ~~~ c^i \in \mathbb{R}
\end{equation*}

\noindent
But, \(1+x^0 \in \mathbb{R}\), so really, the constant can be absorbed into the \(0^{\text{th}}\)-order term.

\begin{equation}
    \implies f + 1 = f
    \label{eq:fabsorbsone}
\end{equation}

\noindent
So, given \(v_p(f+1)\), by (i):

\begin{equation}
    v_p(f + 1) = v_p(f) + v_p(1)
    \label{eq:vpsplit}
\end{equation}

\noindent
Now, using Eq. \ref{eq:fabsorbsone} and Eq. \ref{eq:vpsplit}:

\begin{align*}
    v_p(f+1) &= v_p(f) \\
    \implies v_p(f) + v_p(1) &= v_p(f) \\
    \implies v_p(1) &= v_p(f) - v_p(f) = 0
\end{align*}

\end{proof}

\end{document}
