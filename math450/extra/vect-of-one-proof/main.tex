\documentclass{article}

\usepackage[letterpaper, tmargin=1in]{geometry}
\usepackage{physics}
\usepackage{amsmath, amsfonts, amsthm}
\usepackage{enumerate}

\title{Proof of \(\vb v_p(1) = 0\).}
\author{Chase A. Lotito}

\begin{document}

\maketitle{}

\section{Definiton of a (tangent) vector.}

A (tangent) vector at a point \(p \in M\), where \(M\) is a smooth manifold, is an operator:

\begin{equation*}
    \vb v_p : \mathcal{F}M \to \mathbb{R}
\end{equation*}

\noindent
Our vector \(\vb v_p\), the operator, takes a smooth function (which itself maps our manifold to a scalar quantity) and provides a real number, given these conditions are satisfied:

\begin{enumerate}[i.]
    \item \( \vb v_p (f + g) = \vb v_p(f) + \vb v_p(g) \)
    \item \( \vb v_p (cf) = c (\vb v_p(f)), ~~~ c \in \mathbb{R} \)
    \item \( \vb v_p(fg) = \vb v_p(f)g(p) + f(p)\vb v_p(g) \)
\end{enumerate}

\section{Proof of why \(\vb v_p(1) = 0\)}

From the definition of the vector, we see that it acts like a derivative. Both derivatives and vectors operate on functions and provide a scalar quantity in return. From derivatives, we know that:

\begin{equation*}
    \frac{d}{dx}(1) = 0
    \label{eq:derivative-of-one}
\end{equation*}

\noindent
Can we prove then, that the (tangent) vector operating on unity is also zero?

\begin{proof}[Proof]

    Given a (tangent) vector \(\vb v_p\), let's apply the \textit{Leibniz rule} as it operates on \(1 \in \mathcal{F}M\):

\begin{equation}
    \vb v_p(1) = \vb v_p(1 \cdot 1) \overset{\text{\tiny{(iii)}}}{=} \vb v_p(1) \cdot 1 + 1 \cdot \vb v_p(1) = \vb v_p(1) + \vb v_p(1)
    \label{eq:starteq}
\end{equation}

\noindent
Rearranging (\ref{eq:starteq}):

\begin{align*}
    \vb v_p(1) - \vb v_p(1) = \vb v_p(1) \\
    \leadsto \vb v_p(1) = 0
\end{align*}

\end{proof}

\noindent
Given Condition (ii), \(\vb v_p(1) = 0\) generalizes \(\forall c \in \mathbb{R}\):

\begin{equation*}
    \vb v_p(c) = \vb v_p(c \cdot 1) = c(\vb v_p(1)) = c \cdot 0 = 0
\end{equation*}

%\begin{proof}[Proof]
%
%Let our tangent vector \(v_p\) act on the following smooth function from \(\mathcal{F}M\):
%
%\begin{equation*}
%    f = x^0 + c^1 x^1 + c^2 x^2 + \cdots + c^n x^n, ~~~ c^i \in \mathbb{R}
%\end{equation*}
%
%\noindent
%From \(f\), the first term \(x^0 \in \mathbb{R}\). If we take \(f+1\), then we'd have:
%
%\begin{equation*}
%    f = 1 + x^0 + c^1 x^1 + c^2 x^2 + \cdots + c^n x^n, ~~~ c^i \in \mathbb{R}
%\end{equation*}
%
%\noindent
%But, \(1+x^0 \in \mathbb{R}\), so really, the constant can be absorbed into the \(0^{\text{th}}\)-order term.
%
%\begin{equation}
%    \leadsto f + 1 = f
%    \label{eq:fabsorbsone}
%\end{equation}
%
%\noindent
%So, given \(v_p(f+1)\), by (i):
%
%\begin{equation}
%    v_p(f + 1) = v_p(f) + v_p(1)
%    \label{eq:vpsplit}
%\end{equation}
%
%\noindent
%Now, using Eq. \ref{eq:fabsorbsone} and Eq. \ref{eq:vpsplit}:
%
%\begin{align*}
%    & v_p(f+1) = v_p(f) \\
%    \leadsto ~~ & v_p(f) + v_p(1) = v_p(f) \\
%    \leadsto ~~ & v_p(1) = v_p(f) - v_p(f) = 0
%\end{align*}
%
%\noindent
%Notice, this applies for all constants \(a \in \mathbb{R}\), just factor out \(a\) using (ii).
%
%\end{proof}
%
\end{document}
