\documentclass{article}

\usepackage[T1]{fontenc}
\usepackage{physics}
\usepackage{amsmath, amsfonts, amsthm}
\usepackage{geometry}
\usepackage{titling}
\usepackage{braket}
\usepackage{pgfplots}

\title{Minkowksi Space, \(\mathbb{M}^4\)}
\author{Chase A. Lotito}
\date{\today}

\begin{document}

\maketitle

\section{Introduction}

Given Poincar\'e's and Einstein's budding theory of the relativistic nature to our reality in 1904 and 1905, the Lithuanian mathematician Hermann Minkowski developed the fundamental geometrical theory of spacetime. His work related to spacetime is known as Minkowksi spacetime or Minkowski space, which is a four-dimensional linear space containing information of the spacial dimensions \({x,y,z}\in\mathbb{R}^3\) and the temporal curve \({t}\in\mathbb{R}\). Together the combined spacetime manifold can be written as \(M \cong \mathbb{R}^4 = \mathbb{R}^{3,1} = \mathbb{M}^4 \).

\section{Right to the point}

The quadratic form in Minkowksi space is:

\begin{equation}
    Q_{\mathbb{M}} = x^2 + y^2 +z^2 - c t^2
\end{equation}

\noindent
where we assume three spacial coordinates \( \set{x,y,z} \), a singular temporal coordinate \(\set{t}\), and the speed of light \(c \in  \mathbb{R}\). Together, a single point \(p\in M\) is an \emph{event} in spacetime and is of the form \((t,x,y,z)\).

\smallskip
\noindent
The following dictates that spacetime is fundamentally governed by a hyperbolic geometry. If we shrink the problem down to just a single spacial dimension and a single temporal dimension, so \(M\cong \mathbb{R}^2\) where \(\set{x,t}\). The quadriatic form is \(Q_{\mathbb{M}}=x^2-t^2\).


\end{document}