\documentclass{article}
\usepackage[letter, margin=1in, top=0in]{geometry}
\usepackage{amsmath}
\usepackage{amssymb}
\usepackage{physics}

\title{MATH450: Basic Preliminaries}
\author{Chase Lotito}
\date{}

\begin{document}

\maketitle

\section*{Vector Space Axioms}

A \textbf{vector space} (a finite dimensional linear space) \(V\) is a group of objects called \textit{vectors} which follow the following rules. Given \(\vb{u}, \vb{v}, \vb{w} \in V\) and \(a,b \in \mathbb{R}\):

% contains the 8 axioms of a linear/vector space
\begin{enumerate}
    \item \( \vb{u} + ( \vb{v} + \vb{w} ) = ( \vb{u} + \vb{v} ) + \vb{w} \)
    \item \( \vb{u} + \vb{v} = \vb{v} + \vb{u} \)
    \item \( \vb{v} + \vb{0} = \vb{v} \)
    \item \( \vb{v} + - \vb{v} = \vb{0} \)
    \item \( a(b\vb{v}) = (ab) \vb{v}  \)
    \item \( 1 \vb{v} = \vb{v}  \)
    \item \( a (\vb{u} + \vb{v}) = a\vb{u} +a\vb{v} \)
    \item \( (a + b) \vb{v} = a\vb{v} + b\vb{v} \)
\end{enumerate}

\noindent
Note again, a linear space \(L\) follows the same axioms, but it can be infinite dimensional.

\section*{Abstract Algebra Topics}

The \textbf{Cartesian Product}, for the sets \(A\) and \(B\), is the set of ordered pairs:

\begin{equation*}
    A \times B = \{ (a,b) : a \in A, b \in B  \}
\end{equation*}

\noindent
The cartesian product is not associative and it is not commutative.

\section*{Manifolds}

A \textbf{manifold} \(M\) is a smooth/differentiable space that is locally Euclidean.

\end{document}
