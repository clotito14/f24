\documentclass{article}

\usepackage{tgschola}
\usepackage{graphicx}
\usepackage{amsmath}
\usepackage{amsfonts}
\usepackage[margin=1in]{geometry}
\usepackage{enumerate}          
\usepackage[shortlabels]{enumitem}
\usepackage{geometry}
\usepackage{titling}
\usepackage{setspace}

\setlength{\droptitle}{-6em}     % Eliminate the default vertical space

\title{\textbf{\textit{Dieselgate}: Volkswagen Emissions Scandal} \\ {\large ECE495E Ethics Paper}}
\author{Chase A. Lotito, B.S. E.E., SIUC COECTM}
\date{\today}

\linespread{1.5}

\begin{document}

\maketitle

% Introduction

With the climate change movement, governments are implementing regulations on vehicles to prevent any one car from polluting above a certain threshold. For example, [almost all] cars in the United States need to be ``emissions tested" every two years to make sure they are up to code. Car manufacturers have to ensure their cars continue to meet emissions standards from around the world, or else be unable to sell cars in that market. Unless that company can outsmart the system. In 2015, Volkswagen was found to be implementing emissions test cheats called "defeat devices" to their line of clean diesel cars, since they dumped 40 times more emissions than allowed \cite{engineer-pleads-guilty}. James Liang, the Volkswagen engineer behind the defeat device, plead guilty on September 9th, 2016 for conspiracy to defraud, wire fraud, and violating the Clean Air Act based on his involvement in Dieselgate \cite{engineer-pleads-guilty}. 

% (1)
% What are the morally relevant facts of the case? What happened? Why did it happen?
% Who made the key decisions? Who was affected? What were the long-term consequences?

The Environmental Protection Agency (EPA) uncovered the Volkswagen ``defeat devices'' back in 2015, initially finding them in 482,000 cars in the U.S., only for Volkswagen to admit to adding the cheat to 11,000,000 cars worldwide \cite{bbc-russell}. The cause of this goes back to 2006 when Liang---Leader of Diesel Competence---and his team of engineers in the Diesel Development Department were unable to design the EA-189 diesel engine to U.S. emissions standards \cite{engineer-pleads-guilty}. Instead of submitting to the limitations of the technology, the Volkswagen engineers designed a work-around in the diesel cars to under-report emissions when it could sense it was being tested. The engineering team, under the leadership of James Liang, even went as far as to lie to the EPA and other U.S. regulators in meetings regarding the performance of the VW diesel cars \cite{engineer-pleads-guilty}. Ultimately, this cost Liang his reputation and he had to face legal consequences, some Volkswagen executives left to save face, the company itself faced billions in fines and a hurt reputation, and everyone around the World was put at a heightened risk of pollution. 

% (2)
% What is the most important moral issue raised by the case?

The moral issue in this case is should an engineer fabricate product performance in order to design a product for their company to profit from? In the case of Liang and his team, they might have felt pressured to provide a working design when the diesel engine itself could not be optimized any further. Liang broke nearly all the points in the IEEE Code of Ethics (I, IV, V, IX, X) \cite{ieee-ethics} by pushing a non-sustainable engine, breaking the law on multiple counts, providing false data on the engine, tarnishing his and others reputations, and involving other engineers in this scandal. Instead, Liang should have gathered sufficient evidence that designing a sustainable version of the EA-189 engine was impossible and present that to the executives managing the project. While it loses the Volkswagen corporation money in unfruitful R\&D, the public deserves safe designs under the IEEE Code of Ethics. Although, this viewpoint is easy to say but hard to show in practice given the immense power of corporations to manipulate individuals for its own gain.

% Conclusion

Overall, the Dieselgate scandal elucidates how important it is to subscribe to a solid set of ethics even when it means a design project needs to be shut down. The welfare of the public has to be paramount to the welfare of corporations, especially corporations that depend on the intellectual work of engineers. When given responsibility, like the responsibility given to James Liang, it has to be taken seriously, when at the end of the day millions of people will be affected by how environmentally-friendly your product is. If instead of taking the high road, if we engineers decide to skirt around what is right with tricks, society will only fall deeper into despair. 

% References
\bibliographystyle{plain}
\bibliography{bib.bib}

\end{document}