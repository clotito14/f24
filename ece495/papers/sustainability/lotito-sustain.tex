\documentclass{article}

\usepackage{tgschola}
\usepackage{graphicx}
\usepackage{amsmath}
\usepackage{amsfonts}
\usepackage[margin=1in]{geometry}
\usepackage{enumerate}          
\usepackage[shortlabels]{enumitem}
\usepackage{geometry}
\usepackage{titling}


\setlength{\droptitle}{-6em}     % Eliminate the default vertical space

\title{\textbf{Climate Solution: LED Lighting} \\ {\large ECE495E Sustainability Paper}}
\author{Chase A. Lotito -- \textit{SIUC Undergraduate}}
\date{\today}

\begin{document}

\maketitle

% 1. Briefly describe solution and what sector uses said solution.

The ``LED Lighting'' solution from \textit{drawdown.org}, belonging to the Electricity sector, could lead to a 15.69 gigaton reduction in \(CO_2\) emissions by 2050 \cite{drawdown}. LED Lighting pertains to changing all lighting technologies, like incandescent and halogen lightbulbs, to light-emitting diode (LED) bulbs. The main benefit to using a semiconductor technology like LEDs is their high efficiency as compared to more traditional technologies. LEDs leverage the light emitted at a set frequency from electrons changing states in a semiconductor, while incandescent solutions leverage the light emitted in a spectrum from a heated filament. The main reason incandescent bulbs are so inefficient, is since the majority of energy applied to them is spent as light in the infrared band, which is invisible to the human eye (felt as heat). On the other hand, LEDs are so efficient since all nearly all of the energy applied to them is converted into light \textit{at a specific wavelength in the visible band} (or ultraviolet, if designed for that).

% 2. Identify key challenges to society adopting solution

The key challenges to implementing LEDs in place of other lighting solutions are: reducing LED costs and educating others about LEDs. The biggest impact of these solutions would be seen in third-world countries who have less access to LED technologies due to larger upfront costs and access to already well-established incandescent or halogen technologies. The analysis done by Drawdown researchers has a full-LED implementation in the United States has the upfront cost of \$2.16 trillion USD \cite{drawdown}. Which upfront, is a very large cost, but that is trounced by the net savings which are \$4.47 trillion USD over the lifetime of the LEDs \cite{drawdown}.

% 3. What could engineers do to help mitigate challenges and move sol. forward?

In order to make LEDs more accessible to others, engineers have to find ways to make LED technologies cheaper, more specifically, the technologies built around the LED. Industrial engineers could work on reducing the manufacturing costs, which at economies of scale would reduce the selling price. Also, engineers could do better at teaching consumers about the benefits of LEDs as compared to the costs of other technologies, which could marginally help increase the demand for LEDs. Also, engineers can always work at improving the longevity of LEDs to further increase the net savings and make the upfront cost to some consumers less daunting.

% 4. Global, economic, environmental, societal impact of solution?

Globally, a move towards LED technologies would put a large emphasis on semiconductor research, a field that would have impacts on other clean energy fields like solar energy and power electronics. Economically, more LEDs in the market, pushing the price down, would shift more capital into the semiconductor industry; a stronger semiconductor industry in the United States would create millions of jobs and reduce the United State's reliance on semiconductors from overseas (this would add another source of semiconductors to the global supply chain which is good for everyone). In society, a shift to LEDs, which has already been a movement for many years, can show the benefits and drawbacks of technologies we use everyday which can create smarter consumers. 

% 5. What did you learn or are surprised by about this solution?

What's surprising is how much \(CO_2\) can be saved by changing something as simple as a lightbulb. Marginal emissions by simple everyday technologies have impacts on everything on Earth, and it's up to engineers to make sure that the technologies we implement, even if small, are as efficient as we can make them. Inefficiency at scale is devastating. 

% References
\bibliographystyle{plain}
\bibliography{bib.bib}

\end{document}